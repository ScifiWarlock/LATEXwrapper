\documentclass{article}
\title{Multi 13.6 Notes}
\author{Omkar Tasgaonkar}
\usepackage{graphicx}
\graphicspath{{/Users/omkar/Desktop/Screenshots/}}

\begin{document}
\maketitle
\tableofcontents
\listoffigures
\pagebreak

\subsection{The Directional Derivative:}

The directional derivative tells us the slope in a direction:\\

\begin{equation}
\frac{\partial f}{\partial x} =  slope\ in\ x-dir
\end{equation}

\begin{equation}
\frac{\partial f}{\partial y} =  slope\ in\ y-dir
\end{equation}

\begin{figure}[h]
\caption{A graph z = f(x, y) represents the hill}
\centering
\includegraphics[width=5cm, height=5cm]{ss 2024-12-24 at 20.55.png}
\end{figure}
We want to find the slope in a certain direction, a rotation of theta in cylindrical

\begin{figure}[h]
\caption{Rotation of theta}
\centering
\includegraphics[width=5cm, height=5cm]{ss 2024-12-24 at 20.57.png}
\end{figure}
So our direction vector $u$
Therefore our directional derivative is the dot product of two vectors, now which vectors?\\

\subsection{The Gradient}

The gradient tells us the biggest derivative vector at a certain point in the graph!

\begin{figure}[h]
\caption{The Gradient, max growth}
\centering
\includegraphics[width=5cm, height=5cm]{ss 2024-12-24 at 21.01.png}
\end{figure}

\begin{equation}
\nabla f(x,y) = \frac{\partial f}{\partial x}\vec{i}+\frac{\partial f}{\partial y}\vec{j}
\end{equation}
So now, if we rotate this gradient by an angle theta, then we can say that the gradient is a subset of rotated gradients
Or that $\nabla$ f(x, y) $\in$ G where G is the superset\\

For the final step, the directional derivative must be a scalar quantity as it is a slope, so:

\begin{equation}
Directional\ derivative = \nabla f(x,y)\cdot u
\end{equation}
Whereby $u$ is the direction vector for angle $\theta$\\


\subsubsection{Using $\nabla$ f(x, y) to find Directional Derivative}
Find the directional derivative of $f(x,y) = 3x^2 - 2y^2 at (-3/4, 0)$ in the direction of:
$3/4\vec{i}+\vec{j}$\\

\begin{equation}
\nabla f(x,y) = 6x\vec{i}-4y\vec{j}
\end{equation}

\begin{equation}
u = 3/5\vec{i}+4/5\vec{j}
\end{equation}

\begin{equation}
\nabla f(-3/4, 0)\cdot u = -9/2\vec{i}\cdot(3/5\vec{i}+4/5\vec{j})
\end{equation}
So the directional derivative or dot product for this question evaluates to: -27/10\\
\end{document}