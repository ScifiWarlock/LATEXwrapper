\documentclass{article}
\title{TA Problems of the Week: Semester One}
\author{Omkar Tasgaonkar}
\usepackage{graphicx}
\usepackage{amsmath}
\graphicspath{{/Users/omkar/Desktop/Screenshots/}}

\begin{document}
\maketitle
\tableofcontents
\pagebreak

\subsection{Problem of the week (9/30 - 10/3): Integrals!}
Given a function $f(x)$ at $x$ = 2$\pi$, which outputs -1:\\


\begin{equation}
If\ f'(x) = -(1 - \sin^{2}(\pi/2 - x))^{1/2}
\end{equation}
Find $f(x)$\\


\subsubsection{Solution:}
Assume domain restriction from 0 to $\pi$/2\\


\begin{equation}
\sin^{2}(\pi/2 -x) = \cos^{2}(x)
\end{equation}

\begin{equation}
f'(x) = -(1 - \cos^{2}(x))^{1/2}
\end{equation}

\begin{equation}
-(1 - \cos^{2}(x))^{1/2} = -(\sin^{2}(x))^{1/2}
\end{equation}

\begin{equation}
-(\sin^{2}(x))^{1/2} = -\sin(x)
\end{equation}
Now that we have simplified the derivative to a simple cos function, we can integrate it. Let's review the fundamental theorem of calculus first: \\
\(\int_{a}^{b} g(x) \,dx \) = f(b) - f(a)\\

\[\int -\sin(x) \,dx \]

\begin{equation}
 = \cos(x) + C
\end{equation}

\begin{equation}
\cos(2\pi) + C = -1
\end{equation}

\begin{equation}
C = -2
\end{equation}

\begin{equation}
So\ f(x) = \cos(x) - 2
\end{equation}
\end{document}